\begin{sepframe}{Functional code}
    {}
\end{sepframe}

\begin{frame}
    \frametitle{Functional code}
    \framesubtitle{Tips and tricks}

    \begin{itemize}[<+->]
        \item Make your objects \textbf{immutable}
        \item Make your properties \textbf{private} and \textbf{readonly}
        \item Make sure your functions are \textbf{total}
    \end{itemize}
\end{frame}

\begin{frame}
    \frametitle{Functional code}
    \framesubtitle{Tips and tricks}

\lstinputlisting[caption={Object immutability}]{src/session/functional/resources/immutability1-a.php}
\pause
\lstinputlisting[caption={Object immutability}]{src/session/functional/resources/immutability1-b.php}
\end{frame}

\begin{frame}
    \frametitle{Functional code}
    \framesubtitle{Tips and tricks}

\lstinputlisting[caption={Object immutability}]{src/session/functional/resources/immutability2-a.php}
\pause
\lstinputlisting[caption={Object immutability}]{src/session/functional/resources/immutability2-b.php}
\end{frame}

\begin{frame}
    \frametitle{Functional code}
    \framesubtitle{Tips and tricks}

\lstinputlisting[caption={Object immutability}]{src/session/functional/resources/properties1-a.php}
\pause
\lstinputlisting[caption={Object immutability}]{src/session/functional/resources/properties1-b.php}
\end{frame}

\begin{frame}
    \frametitle{Functional code}
    \framesubtitle{Tips and tricks}

\lstinputlisting[caption={Object immutability}]{src/session/functional/resources/total1-a.php}
\pause
\lstinputlisting[caption={Object immutability}]{src/session/functional/resources/total1-b.php}
\end{frame}

\begin{frame}
    \frametitle{Functional code}
    \framesubtitle{Tips and tricks}

    A total function is a function which is defined for all inputs.

    \pause

    \begin{definition}[Total function]
        \begin{align*}
        & f: A \mapsto B\\
        & \forall a \in A \implies \exists f(a) \in B
        \end{align*}
    \end{definition}
\end{frame}

\begin{frame}
    \frametitle{Functional code}
    \framesubtitle{Tips and tricks}

    A partial function is a function which is not defined for all inputs.

    \pause

    \begin{definition}[Partial function]
        \begin{align*}
        & f: A \mapsto B\\
        & \exists x \in A \implies \not\exists f(x) \in B
        \end{align*}
    \end{definition}
\end{frame}

\begin{frame}
    \frametitle{Functional code}
    \framesubtitle{Tips and tricks}

    \begin{definition}[Total function]
        \begin{align*}
        & f: A \mapsto B\\
        & \forall a \in A \implies \exists f(a) \in B
        \end{align*}
    \end{definition}

    \begin{definition}[Partial function]
        \begin{align*}
        & f: A \mapsto B\\
        & \exists x \in A \implies \not\exists f(x) \in B
        \end{align*}
    \end{definition}
\end{frame}

\begin{frame}
    What\pause\ the\pause\ link\pause\ ????
\end{frame}

\begin{frame}
    \frametitle{Functional code}
    \framesubtitle{Tips and tricks}

    \lstinputlisting[caption={A Doctrine repository definition}]{src/session/functional/resources/doctrine-repository1-a.php}
    \pause
    \lstinputlisting[caption={A Doctrine repository in use}]{src/session/functional/resources/doctrine-repository1-b.php}
    % We could talk about throwing instead, but the issue remains
    % the same.
\end{frame}

\begin{frame}
    \frametitle{Functional code}
    \framesubtitle{Tips and tricks}

    \lstinputlisting[caption={Alternative A}]{src/session/functional/resources/square-root1-a.php}
    \pause
    \lstinputlisting[caption={Alternative B}]{src/session/functional/resources/square-root1-b.php}
    \pause
    \lstinputlisting[caption={Alternative C}]{src/session/functional/resources/square-root1-c.php}
\end{frame}

\begin{frame}
    \frametitle{Functional code}
    \framesubtitle{Tips and tricks}

    \begin{itemize}[<+->]
        \item A: Mixing types implies more code for the end user.
        \item B: Great power means great responsibility, has the power to stop the program!
        \item C: Prone to incertitude when it returns 0.
    \end{itemize}
\end{frame}

\begin{frame}
    \frametitle{Functional code}
    \framesubtitle{Tips and tricks}

    What is our best option?

    \pause

    \begin{itemize}[<+->]
        \item Avoid using strict types ?
        \item Wrap the incertitude in a well-know object?
        \item Not using PHP ?
    \end{itemize}
\end{frame}

\begin{frame}
    How about the idea of abstracting the incertitude by wrapping in a well-know
    object?
    \pause
    An object that can be \textbf{maybe} null or \textbf{maybe} the expected
    value?
\end{frame}

\begin{frame}
    \lstinputlisting[caption={Draft Maybe class}]{src/session/functional/resources/maybe-draft1-a.php}
\end{frame}

\begin{frame}
    \lstinputlisting[caption={Draft Maybe class}]{src/session/functional/resources/maybe-draft1-b.php}
\end{frame}

\begin{frame}
    \lstinputlisting[caption={Draft Maybe class}]{src/session/functional/resources/maybe-draft1-c.php}
\end{frame}

\begin{frame}
    \lstinputlisting[caption={Draft Maybe class}]{src/session/functional/resources/maybe-draft1-d.php}
    % Problem is still there, we have to write pure function and
    % avoid mixed types.
\end{frame}

\begin{frame}
    \lstinputlisting[caption={Draft Maybe class}]{src/session/functional/resources/maybe-draft1-e.php}
    % Problem is still there, we have to write pure function and
    % avoid mixed types.
\end{frame}

\begin{frame}
    \lstinputlisting[caption={Draft Maybe class}]{src/session/functional/resources/maybe-draft1-f.php}
    % Problem is still there, we have to write pure function and
    % avoid mixed types.
\end{frame}
