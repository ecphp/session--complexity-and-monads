\begin{sepframe}{Functional code}
    {}
\end{sepframe}

\begin{frame}
    \frametitle{Functional code}
    \framesubtitle{Tips and tricks}

    \begin{itemize}[<+->]
        \item Make your objects \textbf{immutable}
        \item Make your properties \textbf{private} and \textbf{readonly}
        \item Make sure your functions are \textbf{total}
    \end{itemize}
\end{frame}

\begin{frame}
    \frametitle{Functional code}
    \framesubtitle{Tips and tricks}

\lstinputlisting[caption={Object immutability}]{src/session/functional/resources/immutability1-a.php}
\pause
\lstinputlisting[caption={Object immutability}]{src/session/functional/resources/immutability1-b.php}
\end{frame}

\begin{frame}
    \frametitle{Functional code}
    \framesubtitle{Tips and tricks}

\lstinputlisting[caption={Object immutability}]{src/session/functional/resources/immutability2-a.php}
\pause
\lstinputlisting[caption={Object immutability}]{src/session/functional/resources/immutability2-b.php}
\end{frame}

\begin{frame}
    \frametitle{Functional code}
    \framesubtitle{Tips and tricks}

\lstinputlisting[caption={Object immutability}]{src/session/functional/resources/properties1-a.php}
\pause
\lstinputlisting[caption={Object immutability}]{src/session/functional/resources/properties1-b.php}
\end{frame}

\begin{frame}
    \frametitle{Functional code}
    \framesubtitle{Tips and tricks}

\lstinputlisting[caption={Object immutability}]{src/session/functional/resources/total1-a.php}
\pause
\lstinputlisting[caption={Object immutability}]{src/session/functional/resources/total1-b.php}
\end{frame}

\begin{frame}
    \frametitle{Functional code}
    \framesubtitle{Tips and tricks}

    A total function is a function which is defined for all inputs.

    \pause

    \begin{definition}[Total function]
        \begin{align*}
        & f: A \mapsto B\\
        & \forall a \in A \implies \exists f(a) \in B
        \end{align*}
    \end{definition}
\end{frame}

\begin{frame}
    \frametitle{Functional code}
    \framesubtitle{Tips and tricks}

    A partial function is a function which is not defined for all inputs.

    \pause

    \begin{definition}[Partial function]
        \begin{align*}
        & f: A \rightharpoonup B\\
        & \exists x \in A \implies \not\exists f(x) \in B
        \end{align*}
    \end{definition}
\end{frame}

\begin{frame}
    \frametitle{Functional code}
    \framesubtitle{Tips and tricks}

    \begin{definition}[Total function]
        \begin{align*}
        & f: A \mapsto B\\
        & \forall a \in A \implies \exists f(a) \in B
        \end{align*}
    \end{definition}

    \begin{definition}[Partial function]
        \begin{align*}
        & f: A \mapsto B\\
        & \exists x \in A \implies \not\exists f(x) \in B
        \end{align*}
    \end{definition}
\end{frame}

\begin{frame}
    % TODO: Total functions are definitely less trouble makers than partials.
    \frametitle{Functional code}
    \framesubtitle{Tips and tricks}

    \begin{itemize}
        \item Total:\pause
        \\\lstinline[language=PHP]!function add (int $a, int $b): int!
        \pause
        \item Partial:\pause
        \\\lstinline[language=PHP]!function divide(int $a, int $b): float!
    \end{itemize}
\end{frame}

\begin{frame}
    What\pause\ the\pause\ link\pause\ ????
\end{frame}

\begin{frame}
    \frametitle{Functional code}
    \framesubtitle{Handling errors}

    \lstinputlisting[caption={A Doctrine repository definition}]{src/session/functional/resources/doctrine-repository1-a.php}
    \pause
    \lstinputlisting[caption={A Doctrine repository in use}]{src/session/functional/resources/doctrine-repository1-b.php}
    % We could talk about throwing instead, but the issue remains
    % the same.
\end{frame}

\begin{frame}
    \frametitle{Functional code}
    \framesubtitle{Tips and tricks}

    \lstinputlisting[caption={Return null}]{src/session/functional/resources/square-root1-a.php}
    \pause
    \lstinputlisting[caption={Throw exception}]{src/session/functional/resources/square-root1-b.php}
    \pause
    \lstinputlisting[caption={Return sentinel}]{src/session/functional/resources/square-root1-c.php}
\end{frame}

\begin{frame}
    \frametitle{Functional code}
    \framesubtitle{Tips and tricks}

    \begin{itemize}
        \item Return null?\pause
            \\\textcolor{ecgrey!50}{Mixing types implies more code for the end
            user, prevent composition.}
        \item Throw exceptions?\pause
            \\\textcolor{ecgrey!50}{Great power means great responsibility, has
            the power to stop the program.}
        \item Return sentinel?\pause
            \\\textcolor{ecgrey!50}{Prone to incertitude when it returns 0.}
    \end{itemize}
\end{frame}

\begin{frame}
    \frametitle{Functional code}
    \framesubtitle{Return null?}
    % TOSAY: Java, C implements nullable objects by default.
    \begin{itemize}[<+->]
        \item Universal error value that doesn't carry any relevant information
            in case of issues.
        \item Widespread, implemented in most languages but controversial
        \item "the billion-dollar mistake"
        \item Considered as a code smell
        \item Prevent composition pattern
    \end{itemize}
    % TOSAY: Ce dernier point fait la liaison avec le slide suivant.
\end{frame}

\begin{frame}
    \frametitle{Functional code}
    \framesubtitle{Throw exceptions?}
    \begin{itemize}[<+->]
        \item Indicate something unforeseen by the developer
        \item Expensive, destructive
        \item Definitely prevent composition pattern
    \end{itemize}
\end{frame}

\begin{frame}
    \frametitle{Functional code}
    \framesubtitle{Return sentinel?}
    \begin{itemize}[<+->]
        \item Potentially prevent composition pattern
        \item Must be carefully done, each type has its own unit type
        \begin{itemize}[<+->]
            \item \lstinline[language=PHP]!function divide(int $a, int $b): float|string!
            \item \lstinline[language=PHP]!function divide(int $a, int $b): float|array!
            \item \lstinline[language=PHP]!function divide(int $a, int $b): float|null!
            \item \lstinline[language=PHP]!function divide(int $a, int $b): float|int!
        \end{itemize}
        \item In PHP, this is sadly happening
        % Ask the audience in which PHP function this happens ? (strpos)
    \end{itemize}
\end{frame}

\begin{frame}
    \frametitle{Functional code}
    \framesubtitle{What is our best option?}

    \begin{itemize}[<+->]
        \item Not using PHP ?
        \item Avoid using strict types ?
        \item Wrap the incertitude in a well-know object?
    \end{itemize}
\end{frame}

\begin{frame}
    \begin{figure}
        \begin{overprint}
        \onslide<1>
            \begin{center}
                \includegraphics[width=8em]{src/session/functional/resources/cat.png}
            \end{center}
        \onslide<2->
            \begin{center}
                \includegraphics[width=10em]{src/session/functional/resources/box.png}
            \end{center}
        \end{overprint}
    \end{figure}

    \pause
    \pause

    How about the idea of abstracting the incertitude by creating an object that
    would wrap a value in a well-know object?\\

    \pause

    \vspace{1em}
    An object that can be \textbf{maybe} null or \textbf{maybe} the expected
    value?
    % TOSAY: Explain the maybe monad with the S'cat box.
\end{frame}

\begin{frame}
    \lstinputlisting[caption={Draft Maybe class}]{src/session/functional/resources/maybe-draft1-a.php}
\end{frame}

\begin{frame}
    \lstinputlisting[caption={Draft Maybe class}]{src/session/functional/resources/maybe-draft1-b.php}
\end{frame}

\begin{frame}
    \lstinputlisting[caption={Draft Maybe class}]{src/session/functional/resources/maybe-draft1-c.php}
\end{frame}

\begin{frame}
    \lstinputlisting[caption={Draft Maybe class}]{src/session/functional/resources/maybe-draft1-d.php}
    % Problem is still there, we have to write pure function and
    % avoid mixed types.
\end{frame}

\begin{frame}
    \lstinputlisting[caption={Draft Maybe class}]{src/session/functional/resources/maybe-draft1-e.php}
    % Problem is still there, we have to write pure function and
    % avoid mixed types.
\end{frame}

\begin{frame}
    \lstinputlisting[caption={Draft Maybe class}]{src/session/functional/resources/maybe-draft1-f.php}
    % Problem is still there, we have to write pure function and
    % avoid mixed types.
\end{frame}


\begin{frame}
    \frametitle{Functional code}
    \framesubtitle{The Maybe monad}

    \begin{itemize}[<+->]
        \item Focus on the happy scenario
        \item Return null when something goes wrong
        \item Unable to customize the value when something is wrong
        \item Back to square 1.
    \end{itemize}
\end{frame}

\begin{frame}
    Maybe we could customize the return value in case of something wrong happen?\\
    \vspace{1em}
    \pause
    While still being able to decide what to do when an error happen?
\end{frame}

\begin{frame}
    \lstinputlisting[caption={Draft Either class}]{src/session/functional/resources/either-draft1-f.php}
\end{frame}
