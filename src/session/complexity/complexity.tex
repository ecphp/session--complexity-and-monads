\begin{sepframe}{Complexity}
  {}
\end{sepframe}

\begin{frame}
  \frametitle{Complexity}
  \framesubtitle{Complexity at a glance}

  \begin{itemize}[<+->]
    \item Used to compare algorithms efficiency,
    \item Often calculated using the worst case scenario (usually \textit{when the input grow infinitely})
    \item In ``Big $\mathcal{O}$'' notation: $\mathcal{O}(1)$, $\mathcal{O}(n)$, $\mathcal{O}(n \times log(n))$, $\mathcal{O}(n^2)$
  \end{itemize}

  \note[item]{
    Let's make a reminder on what complexity is.
  }

  \note[item]{
    Read the bullet points
  }
\end{frame}

\begin{frame}
  \frametitle{Complexity}
  \framesubtitle{Types}

  \begin{itemize}[<+->]
    \item Time complexity
    \item Space complexity
    \item Kolmogorov complexity
    \item Cyclomatic complexity
  \end{itemize}

  \note[item]{
    How about now diving into the marvellous world of algorithmic
    complexity?
    Worry not! We are just going to scratch the surface, nothing fancy
    at all.
  }
  \note[item]{
    We are going to "explore" 4 complexity types. Here they are.
  }
  \note[item]{
    Lets start with the first one, this is usually the most classical one
    when dealing with algorithms.
  }
\end{frame}

\begin{frame}
  \frametitle{Complexity}
  \framesubtitle{Constant time complexity: $\mathcal{O}(1)$}
  \lstinputlisting{src/session/complexity/resources/time-complexity-constant.php}

  \note[item]{
    Let's explain the time complexity with some examples...
  }
  \note[item]{
    Here's the first one. Can you guess what this snippet is doing?
  }
  \note[item]{
    Well, here's the solution. This snippet is returning the first item
    of any type of iterable. If we had to measure its complexity, it would
    be $\mathcal{O}(1)$
  }
  \note[item]{
    Pay attention, here we are not measuring the time, we are measuring how
    much unit of time on a particular computer, no matter how fast or slow
    it is, it would take to get the
    expected output when the size of the input grows infinitely.
  }
\end{frame}

\begin{frame}
  \frametitle{Complexity}
  \framesubtitle{Linear time complexity: $\mathcal{O}(n)$}
  \lstinputlisting{src/session/complexity/resources/time-complexity-linear.php}
  \note[item]{
    Just by looking at it, can you guess what this algo is doing?
  }
  \note[item]{
    Well, here's the solution. This snippet is returning the first item
    of any type of iterable.
    The algorithm must go through each single one item of the iterable,
    therefore, $\mathcal{O}(n)$ is the complexity.
  }
\end{frame}

\begin{frame}
  \frametitle{Complexity}
  \framesubtitle{Exponential time complexity: $\mathcal{O}(n^2)$}
  \lstinputlisting{src/session/complexity/resources/time-complexity-exp.php}
  \note[item]{
    This algorithm is a bit different, it contains nested loops...
  }
  \note[item]{
    Well, here's the solution. This snippet is returning the sum of all
    items in the matrix.
    The algorithm must go through each single row then each single column,
    therefore, $\mathcal{O}(n^2)$ is the complexity.
  }
\end{frame}

\begin{frame}
  \frametitle{Complexity}
  \framesubtitle{Space complexity}
  \lstinputlisting{src/session/complexity/resources/space-complexity.php}

  \note[item]{
    Let's no focus on the algorithm here. We see that a variable \$data is
    is an array of n elements. We also see that the variable \$sum holds
    one element. Therefore the complexity if $\mathcal{O}(n + 1)$, and then
    when n goes up to infinity, $\mathcal{O}(n)$.
  }
\end{frame}

\begin{frame}
  \frametitle{Complexity}
  \framesubtitle{Kolmogorov complexity}

  The Kolmogorov complexity is the shortest size of a program that yield the
  expected output.

  \note[item]{
    The Kolmogorov complexity, besides its funny name, is something not so
    common in our day to day development.
  }
  \note[item]{
    I wanted to talk about it because I want to show that there are different
    types of complexity but also this might be something you already
    encountered without even knowing that it has a name.
  }
  \note[item]{
    I'm going to explain it by showing an example here under.
  }
\end{frame}

\begin{frame}
  \frametitle{Complexity}
  \framesubtitle{Kolmogorov complexity}

  \begin{scriptsize}
    \begin{itemize}[<+->]
      \item \texttt{1111111111111111111111111111111111111111111111111111111111111111}
      \item \texttt{317b773017df0ab62b15cd3f2ad17d7b13ab02f05f4943011ef8c4067d1ca0a5}
    \end{itemize}
  \end{scriptsize}

  \note[item]{
    Here are two strings of 64 characters each.
  }
  \note[item]{
    Can you guess which one of these strings has the smallest algorithm
    yielding those strings?
  }
  \note[item]{
    Another way to do this, let's say that you're on the phone with your
    friend. Your phone is running out of battery and could shut down at any
    moment. Which one of these strings would be faster and easier to
    describe to your friend so he can write it down?
  }
\end{frame}

\begin{frame}
  \frametitle{Complexity}
  \framesubtitle{Kolmogorov complexity}

  \lstinputlisting{src/session/complexity/resources/kolmogorov-complexity-example1.php}
\end{frame}

\begin{frame}[fragile,c]
  \frametitle{Complexity}
  \framesubtitle{Kolmogorov complexity}

  \makebox[\linewidth]{\includegraphics[height=.70\paperheight]{src/session/complexity/resources/Screenshot_20220601_090230.png}}
\end{frame}

\begin{frame}
  \frametitle{Complexity}
  \framesubtitle{Kolmogorov complexity}
  \begin{itquote}
    There is no way to tell if the Kolmogorov complexity of an algorithm is
    the shortest one.
  \end{itquote}

  \note[item]{
    The Kolmogorov complexity is the generalization of Shannon's theory of
    information.
  }
  \note[item]{
    It has links to many other fields like with Gödel's incompleteness
    theorem, Turing's halting problem, compression theory, ...
  }

  \note[item]{
    There is no mechanical device to determine the size of the smallest
    program that produces a given string. It is not that our current level
    of computer technology is insufficient for the task at hand, or that we
    are not clever enough to write the algorithm. Rather, it was proven that
    the very notion of description and computation shows that no such
    computer can ever possibly perform the task for every string. While a
    computer might find some pattern in a string, it cannot find the best
    pattern. We might find some short program that outputs a certain
    pattern, but there could exist an even shorter program. We will never
    know.
  }
  \note[item]{
    The proof that the Kolmogorov complexity of a sequence is not computable
    is a bit technical. But it is a proof by contradiction, and we can get a
    sense of how that works by looking at a paradoxe.
    We can prove that a computer will never be able to find the best pattern
    with this little story...
    Let's say we look at naturals numbers and we claim that every single
    natural have interesting properties.
    1 because it's the first number.
    2 because it's the first even number.
    3 because it's the first odd number.
    4 because it's 2+2 or 2x2...
    We can continue very far. At some point let's say we come to a number
    that does not have interesting properties and call it
    "first uninteresting number". But that in itself, is an interesting
    property ! The first uninteresting number is in fact interesting !
  }
\end{frame}

\begin{frame}
  \frametitle{Complexity}
  \framesubtitle{Cyclomatic complexity}

  \lstinputlisting{src/session/introduction/resources/complexity-example.php}

  \note[item]{
    Complexity theory is almost done, here's another interesting complexity.
  }
  \note[item]{
    The CC can be computed by looking at the decision point in the code.
  }
  \note[item]{
    At each decision point, we add +1 when we enter the decision point.
  }
  \note[item]{
    The CC will be the maximum of these.
  }
  \note[item]{
    This Symfony snippet can be represented as a tree.
    You should recognize familiar stuff in it.
  }
\end{frame}

\begin{frame}[fragile,c]
  \frametitle{Complexity}
  \framesubtitle{Cyclomatic complexity}

  \makebox[\linewidth]{\includegraphics[height=.70\paperheight]{src/session/complexity/resources/complexity-example.png}}

  \note[item]{
    The green color represent the happy path, the red one the unhappy path.
  }
  \note[item]{
    Notice here the interogation marks, we'll come back to it later on.
  }
  \note[item]{
    Have you ever asked yourself why sometimes we have to compare a value
    against null ? We will come back to this later.
  }
\end{frame}

\begin{frame}[fragile,c]
  \frametitle{Complexity}
  \framesubtitle{Cyclomatic complexity}

  Conditions and type checks adds complexity to a program, we are going to see
  how we can get rid of them in a nice an clean way.

  \note[item]{
    FYI, The controversial value "null" is a billion dollar mistake
    according to its creator Tony Hoare, the creator of the quicksort algo,
    the null value and ALGOL. In a software conference in 2009, he
    apologized for inventing the null reference!
  }
  \note[item]{
    But let's come back to our subject, how could we get rid of these
    conditions in our snippet... we will come back to the the null value
    later.
  }
\end{frame}

\begin{frame}[fragile,c]
  \Large
  But\pause\ but\pause\ but\pause ... we need conditions !

  \note[item]{
    Of course we need conditions, this is the basically what defines the
    logic. What I mean here, is how could we reduce the way we write our
    code so it contains less conditions while being easy to read and hack.
  }
\end{frame}

\begin{frame}[fragile,c]
  \begin{center}
    \Huge
    Early returns
  \end{center}

  \note[item]{
    Boom ! That's one way to do it.
  }
  \note[item]{
    For those who works with me, they already know what I'm talking about,
    I'm always hammering people to use and abuse them. But why ?
  }
  \note[item]{
    Early returns allows you to have a clearer view of what you're trying
    to do by ``flattening'' the structure of the code using ``early returns''
    statements. Let's have a look at an example.
  }
\end{frame}

\begin{frame}[fragile,c]
  \frametitle{Complexity}
  \framesubtitle{Cyclomatic complexity}

  \begin{lstlisting}[language=php]
    if ($expr1 && $expr2 && $expr3) {
        return true;
    }

    return false;
    \end{lstlisting}

  \pause

  is equivalent to:

  \begin{lstlisting}[language=php]
    if (! $expr1) {
        return false;
    }

    if (! $expr2) {
        return false;
    }

    if (! $expr3) {
        return false;
    }

    return true;
    \end{lstlisting}

  \note[item]{
    I guess you already had such pattern in your code right?
    The relevant part of the code or the happy path is wrapped in a huge
    condition expression, or even worst. And then, the last line of your
    function is the unhappy path. I don't like that. It's not straighforward
    when skimming the code to find what you're looking for or to find what
    is the return value of an algorithm.
  }
  \note[item]{
    Using early returns let you to split the huge condition expression
    into multiple small conditions.
  }
  \note[item]{
    Those early returns are inverted so we can bail out as soon as we
    encounter something wrong.
  }
  \note[item]{
    Notice also that the happy path is usually after all these ``guards''.
  }
  \note[item]{
    Usually, by looking at the last line of a function, method or algorithm,
    you know what it is returning in case of successful execution.
  }
  \note[item]{
    It is often easier to understand than having a huge logical expression
    to evaluate.
  }
  \note[item]{
    Let you customize a precise custom error value per condition.
  }
  \note[item]{
    Let's apply this trick to our Symfony snippet now...
  }
\end{frame}

\begin{frame}
  \frametitle{Complexity}
  \framesubtitle{Cyclomatic complexity}

  \lstinputlisting{src/session/complexity/resources/complexity-example-early-returns.php}
\end{frame}

\begin{frame}
  \frametitle{Complexity}
  \framesubtitle{Cyclomatic complexity}

  \begin{itemize}[<+->]
    \item Think to the ``\textcolor{red}{unhappy}'' paths at first
    \item The ``\textcolor{green}{happy}'' path, is usually the last line,
    \item Easier to read, understand,
    \item Longer to write.
  \end{itemize}
\end{frame}

\begin{frame}[fragile,c]
  \begin{center}
    \Huge
    Programming paradigms
  \end{center}
  \note[item]{
    This one is a bit funny but...
  }
  \note[item]{
    Another way to reduce the complexity of your code is to reduce the
    amount of code you write. Obviously.
  }
  \note[item]{
    Indeed. Some paradigms make your code more verbose and much prone to
    issues by creating unecessary variables and statements.
  }
\end{frame}

\begin{frame}
  \frametitle{Complexity}
  \framesubtitle{Imperative programming}

  Imperative programming tells the machine how to do something.\pause
  (resulting in what you want to happen).

  \note[item]{
    Some examples: PHP, Java, C, Python...
  }
\end{frame}

\begin{frame}
  \frametitle{Complexity}
  \framesubtitle{Declarative programming}

  Declarative programming tells the machine what you would like to happen.\pause
  (and the computer figures out how to do it)

  \note[item]{
    Some examples: Lambda calculus, Lisp, Scheme, Prolog, Erlang, SQL,
    Ocaml, Haskell,...
  }
\end{frame}

\begin{frame}
  \frametitle{Complexity}
  \framesubtitle{Imperative and declarative programming}

  \lstinputlisting[caption={Imperative programming}]{src/session/history/resources/oop.js}
  \pause
  \lstinputlisting[caption={Declarative programming}]{src/session/history/resources/fp.js}

  \note[item]{
    Here's a snippet written in Javascript. First one in imperative programming,
    second one is the same in declarative programming.
  }
\end{frame}

\begin{frame}
  \frametitle{Complexity}
  \framesubtitle{Imperative and declarative programming}

  \lstinputlisting[caption={Imperative programming}]{src/session/complexity/resources/imperative1.js}
  \pause
  \lstinputlisting[caption={Declarative programming}]{src/session/complexity/resources/declarative1.js}

  \note[item]{
    Another example here.
  }
\end{frame}

\begin{frame}[fragile,c]
  \begin{center}
    \Huge
    NIH vs PFE
  \end{center}
\end{frame}

\begin{frame}
  \frametitle{Complexity}
  \framesubtitle{Abstracting away the complexity?}

  Have you every wondered why there are so much packages existing for a
  particular languages?\\
  \pause
  Most probably they were created to fix particular and recurrent issues or
  boring problematic?\\
  \pause
  Sometimes it could be nice to query the language package managers to find if
  the solution to your problem doesn't already exists!\\
  \pause
  This would delegate the responsibility somewhere else and reduce the overall
  complexity of your application.\\
  \pause
  It is also less code to maintain and thus, less prone to bugs.

  \note[item]{
    Another way to get rid of complexity is to avoid the NIH syndrome... What is
    this?

    Basically it refers to the belief that ``in-house'' developments are better
    suited, more secure, more controlled, quicker to develop and incur lower
    overall costs than using existing implementations.
  }
\end{frame}

\begin{frame}
  \frametitle{Complexity}
  \framesubtitle{Confidence in other packages?}

  Relying on someone else's code means a lot, for some reason people prefer
  redoing things on their own.(NIH vs PFE)\\
  \pause
  \vfill
  I believe that developers should be able to evaluate the trust into a
  package based on some key indicators.\\
  \pause
  \vfill
  Tests, popularity, readability, extensibility, practices...
\end{frame}

\begin{frame}
  \frametitle{Complexity}
  \framesubtitle{Confidence in other packages?}

  This is the reason why when making a package, it's better to be as strict
  as possible so the developer using your code doesn't have any surprise when
  using your code in its own.\\
  \pause
  \vfill
  Avoid mixing types, throw proper exceptions in case of issues, etc etc.\\
  \pause
  \vfill
  Let's remind to the audience how to put those tips in practice.

  \note[item]{
    If we put ourselves in the shoes of a package maintainer, it is
    important to ease the work of developers that are going to use our
    package. We can ease that work by avoiding mixing types, throwing proper
    exceptions, avoiding incertitude...
  }
\end{frame}
